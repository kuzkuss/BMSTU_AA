\chapter*{Заключение}
\addcontentsline{toc}{chapter}{Заключение}

В результате выполнения данной лабораторной работы были рассмотрены алгоритмы сортировки (бусинами, подсчетом, гномьей), построены схемы, соответствующие данным алгоритмам. Можно сделать вывод, что сортировку подсчетом стоит применять для массивов, у которых количество элементов намного превышает диапазон значений. Сортировка бусинами малоэффективна в программной реализации. Гномья сортировка работает быстрее с отсортированными массивами и медленнее с массивами отсортированными в обратном порядке. Из результатов замеров времени следует, что сортировка подсчетом и гномья сортировка для случайно заполненных массивов работают практически с одинаковой скоростью. При этом в лучшем и худшем случае выигрывает по времени сортировка подсчетом и поэтому рекомендуется к применению. Сортировка бусинами значительно медленнее. 

В рамках выполнения работы цель достигнута: изучены алгоритмы сортировки.

Решены все задачи:
\begin{itemize}
	\item реализован алгоритм сортировки бусинами;
	\item реализован алгоритм сортировки подсчетом;
	\item реализован алгоритм гномьей сортировки;
	\item созданы схемы изучаемых алгоритмов;
	\item оценены трудоёмкости алгоритмов;
	\item определены средства программной реализации алгоритмов;
	\item выполнены замеры процессорного времени работы реализаций алгоритмов сортировки;
	\item проведен сравнительный анализ по времени работы реализаций алгоритмов сортировки;
	\item подготовлен отчета о выполненной лабораторной работе.
\end{itemize}


