\chapter{Технологическая часть}

В данном разделе приведены средства реализации, сведения о модулях программы, листинги кода, тесты.

\section{Средства реализации}

В качестве языка программирования для реализации данной лабора-
торной работы был выбран язык Python \cite{python}. Данный выбор обусловлен наличием массивов и необходимой функциональности для замеров времени.

Замеры времени проводились при помощи функции process\_time\_ns из библиотеки time \cite{python-time}.

\section{Сведения о модулях программы}

Данная программа разбита на следующие модули:

\begin{itemize}
	\item main.py --- файл, содержащий точку входа в программу, в нем происходит вызов алгоритмов;
	\item sort.py --- файл, содержащий алгоритмы сортировки;
	\item array.py --- файл, содержащий функции работы с массивом;
	\item measurements.py --- файл, содержащий функции замеров времени работы алгоритмов;
\end{itemize}

\section{Реализация алгоритмов}

В листингах \ref{lst:bead_sort}--\ref{lst:gnome_sort} представлены реализации алгоритмов сортировки: бусинами, подсчетом, гномьей.
\clearpage
\begin{center}
\captionsetup{justification=raggedright,singlelinecheck=off}
\begin{lstlisting}[label=lst:bead_sort,caption=Алгоритм сортировки бусинами]
def bead_sort(arr):
	tmp = [0] * max(arr)
	
	for i in range(len(arr)):
		for j in range(arr[i]):
			tmp[j] += 1        
	
	for i in range(len(arr)):
		count = 0
		for j in range(len(tmp)):
			if tmp[j] > i:
				count += 1
		arr[len(arr) - i - 1] = count

\end{lstlisting} 
\end{center}

\begin{center}
\captionsetup{justification=raggedright,singlelinecheck=off}
\begin{lstlisting}[label=lst:counting_sort,caption=Алгоритм сортировки подсчетом]
def counting_sort(arr):
	count_arr = [0] * (max(arr) + 1)
	
	for i in range(len(arr)):
		count_arr[arr[i]] += 1
	
	idx = 0
	for i in range(len(count_arr)):
		for _ in range(count_arr[i]):
			arr[idx] = i
			idx += 1
	    
\end{lstlisting}
\end{center}
\clearpage
\begin{center}
\captionsetup{justification=raggedright,singlelinecheck=off}
\begin{lstlisting}[label=lst:gnome_sort,caption=Алгоритм гномьей сортировки]
def gnome_sort(arr):
	i = 1
	while i < len(arr):
		if i == 0 or arr[i - 1] <= arr[i]:
			i += 1
		else:
			arr[i - 1], arr[i] = arr[i], arr[i - 1]
			i -= 1

\end{lstlisting}
\end{center}

\section{Тестирование}

В таблице \ref{tbl:tests} приведены функциональные тесты для алгоритмов сортировки.

\begin{table}[h]
	\begin{center}
		\caption{\label{tbl:tests} Функциональные тесты}
		\begin{tabular}{|c|c|}
			\hline
			Входной массив & Ожидаемый результат \\ 
			\hline
			$[]$ & $[]$\\ \hline
			$[1]$ & $[1]$\\ \hline
			$[1, 2, 3, 4, 5]$ & $[1, 2, 3, 4, 5]$\\ \hline
			$[5, 4, 3, 2, 1]$ & $[1, 2, 3, 4, 5]$\\ \hline
			$[3, 7, 1, 5, 9]$ & $[1, 3, 5, 7, 9]$\\ \hline
			$[5, 3, 5, 6, 3, 3]$ & $[3, 3, 3, 5, 5, 6]$\\ \hline
		\end{tabular}
	\end{center}
\end{table}

При проведении функционального тестирования, полученные результаты работы программы совпали с ожидаемыми. Таким образом, функциональное тестирование пройдено успешно.

\section*{Вывод}

Были реализованы алгоритмы сортировки бусинами, сортировки подсчетом и гномьей сортировки и проведено тестирование их реализаций.