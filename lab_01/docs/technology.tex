\chapter{Технологическая часть}

В данном разделе приведены средства реализации, сведения о модулях программы, листинги кода, тесты.

\section{Средства реализации}

В качестве языка программирования для реализации данной лабора-
торной работы был выбран язык Python \cite{python}. Данный выбор обусловлен тем, что он позволяет реализовывать сложные задачи за короткие сроки за счет наличия большого количества подключаемых библиотек и простоты синтаксиса.

Замеры времени проводились при помощи функции process\_time\_ns из библиотеки time \cite{python-time}.

\section{Сведения о модулях программы}

Данная программа разбита на следующие модули:

\begin{itemize}
	\item main.py - файл, содержащий точку входа в программу, в нем происходит вызов алгоритмов;
	\item levenshtein.py - файл, содержащий алгоритмы нахождения редакционного расстояния Левенштейна;
	\item damerau\_levenshtein.py - файл, содержащий алгоритмы нахождения редакционного расстояния Дамерау-Левенштейна;
	\item matrix.py - файл, содержащий функции работы с матрицами;
	\item measurements.py - файл, содержащий функции замеров времени работы алгоритмов;
\end{itemize}

\section{Листинги кода}

В листингах \ref{lst:recursive}-\ref{lst:dl} представлены реализации алгоритмов нахождения расстояния Левенштейна и расстояния Дамерау-Левенштейна.
\clearpage
\begin{center}
\captionsetup{justification=raggedright,singlelinecheck=off}
\begin{lstlisting}[label=lst:recursive,caption=Матричный алгоритм нахождения расстояния Левенштейна]
def matrix_levenshtein(str1, str2):
	n1 = len(str1) + 1
	n2 = len(str2) + 1
	
	matrix = create_matrix(n1, n2)
	
	for i in range(1, n1):
		for j in range(1, n2):
			d1 = matrix[i][j - 1] + 1
			d2 = matrix[i - 1][j] + 1
			d3 = matrix[i - 1][j - 1] + (0 if str1[i - 1] == str2[j - 1] else 1)
			matrix[i][j] = min(d1, d2, d3)
	
	return matrix[n1 - 1][n2 - 1]

\end{lstlisting} 
\end{center}

\begin{center}
\captionsetup{justification=raggedright,singlelinecheck=off}
\begin{lstlisting}[label=lst:matrix,caption=Рекурсивный алгоритм нахождения расстояния Дамерау-Левенштейна]
def recursive_damerau_levenshtein(str1, str2):
	n1 = len(str1)
	n2 = len(str2)
	
	if n1 == 0 or n2 == 0:
		return max(n1, n2)
	
	d1 = recursive_damerau_levenshtein(str1[:], str2[:-1]) + 1
	d2 = recursive_damerau_levenshtein(str1[:-1], str2[:]) + 1
	d3 = recursive_damerau_levenshtein(str1[:-1], str2[:-1]) + (0 if str1[-1] == str2[-1] else 1)
	
	if n1 > 1 and n2 > 1 and str1[-1] == str2[-2] and str2[-1] == str1[-2]:
		d4 = recursive_damerau_levenshtein(str1[:-2], str2[:-2]) + 1
		return min(d1, d2, d3, d4)
	else:
		return min(d1, d2, d3)
    
\end{lstlisting}
\end{center}
\clearpage
\begin{center}
\captionsetup{justification=raggedright,singlelinecheck=off}
\begin{lstlisting}[label=lst:cache,caption=Матричный алгоритм нахождения расстояния Дамерау-Левенштейна]
def matrix_damerau_levenshtein(str1, str2):
	n1 = len(str1) + 1
	n2 = len(str2) + 1
	
	matrix = create_matrix(n1, n2)
	
	for i in range(1, n1):
		for j in range(1, n2):
			d1 = matrix[i][j - 1] + 1
			d2 = matrix[i - 1][j] + 1
			d3 = matrix[i - 1][j - 1] + (0 if str1[i - 1] == str2[j - 1] else 1)
			if i > 1 and j > 1 and str1[i - 1] == str2[j - 2] and str2[j - 1] == str1[i - 2]:
				d4 = matrix[i - 2][j - 2] + 1
				matrix[i][j] = min(d1, d2, d3, d4)
			else:
				matrix[i][j] = min(d1, d2, d3)
	
	return matrix[n1 - 1][n2 - 1]

\end{lstlisting}
\end{center}
\clearpage
\begin{center}
\captionsetup{justification=raggedright,singlelinecheck=off}
\begin{lstlisting}[label=lst:dl,caption=Рекурсивный алгоритм нахождения расстояния Дамерау-Левенштейна с кешем]
def recursive_damerau_levenshtein_cache(str1, str2, matrix):
	n1 = len(str1)
	n2 = len(str2)
	
	if matrix[n1][n2] == -1:
		if n1 == 0 or n2 == 0:
			matrix[n1][n2] = max(n1, n2)
	
		else:
			d1 = rec_dam_lev_mtrx(str1[:], str2[:-1], matrix) + 1
			d2 = rec_dam_lev_mtrx(str1[:-1], str2[:], matrix) + 1
			d3 = rec_dam_lev_mtrx(str1[:-1], str2[:-1], matrix) +
			(0 if str1[-1] == str2[-1] else 1)
			
			if n1 > 1 and n2 > 1 and str1[-1] == str2[-2] and str2[-1] == str1[-2]:
				d4 = rec_dam_lev_mtrx(str1[:-2], str2[:-2], matrix) + 1
				matrix[n1][n2] = min(d1, d2, d3, d4)
			else:
				matrix[n1][n2] = min(d1, d2, d3)
	
	return matrix[n1][n2]
    
\end{lstlisting} 
\end{center}

\section{Тестирование}

В таблице \ref{tbl:tests} приведены функциональные тесты для алгоритмов вычисления расстояния Левенштейна и Дамерау-Левенштейна.
\\[7\baselineskip]

\begin{table}[!h]
	\begin{center}
        \begin{threeparttable}
        \captionsetup{justification=raggedright,singlelinecheck=off}
		\caption{\label{tbl:tests} Тесты}
		\begin{tabular}{|c|c|c|c|c|}
			\hline
			& \multicolumn{2}{c|}{Входные данные} & \multicolumn{2}{c|}{Ожидаемый результат} \\
			\hline
			№&Строка 1&Строка 2&Левенштейн&Дамерау-Левенштейн\\
			\hline
            1&пустая строка&пустая строка&0&0 \\
            \hline
            2&пустая строка&сон&3&3 \\
            \hline
            3&корабль&пустая строка&7&7 \\
            \hline
            4&мост&мос&1&1 \\
			\hline
			5&кот&коооот&3&3 \\
			\hline
			6&ура&уар&2&1 \\
			\hline
			7&абв&ваб&2&2 \\
			\hline
			8&повод&пвд&2&2 \\
			\hline
		\end{tabular}
        \end{threeparttable}
	\end{center}
\end{table}

При проведении функционального тестирования, полученные результаты работы программы совпали с ожидаемыми. Таким образом, функциональное тестирование пройдено успешно.

\section*{Вывод}

Были реализованы алгоритмы: вычисления расстояния Дамерау-\newlineЛевенштейна с помощью матрицы, рекурсивно, рекурсивно с кешем, а также вычисления расстояния Левенштейна с помощью матрицы.