\chapter*{Введение}
\addcontentsline{toc}{chapter}{Введение}

Расстояние Левенштейна --- это минимальное количество редакторских операций, необходимых для преобразования одной строки в другую. Редакторские операции:
\begin{itemize}
	\item вставка 1 символа;
	\item удаление 1 символа;
	\item замена 1 символа.
\end{itemize}

Преобразовать одно слово в другое можно различными способами, количество действий также может быть разным. При вычислении
расстояния Левенштейна следует выбирать минимальное количество
действий.

Исследования Ф. Дамерау показали, что наиболее частая ошибка
при наборе слова --- случайная перестановка двух соседних букв. В случае одной транспозиции расстояние Левенштейна равно 2. При использовании поправки Дамерау транспозиция принимается за единичное расстояние \cite{levenshtein}. Для расстояния Дамерау-Левенштейна к редакторским операциям, указанным выше добавляется транспозиция --- перестановка двух соседних символов. 

Алгоритмы нахождения расстояния Левенштейна и Дамерау-Левенштейна находят применение в следующих областях:
\begin{itemize}
	\item компьютерная лингвистика (задача автозамены и исправления ошибок);
	\item биоинформатика (задача анализа иммунитета).
\end{itemize}

Целью данной лабораторной работы является изучение методов динамического программирования на материале расстояний Левенштейна и Дамерау-Левенштейна. Для достижения поставленной цели требуется решить следующие задачи:

\begin{enumerate}
	\item Изучение расстояний Левенштейна и Дамерау-Левенштейна;
	\item Создание схем изучаемых алгоритмов;
	\item Реализация алгоритмов поиска расстояний Левенштейна и Дамерау-Левенштейна;
	\item Оценка затрат алгоритмов по памяти;
	\item Определение средств программной реализации выбранных алгоритмов;
	\item Выполнение замеров процессорного времени работы реализаций алгоритмов поиска расстояний Левенштейна и Дамерау-Левенштейна;
	\item Проведение сравнительного анализа по времени нерекурсивных алгоритмов поиска расстояний Левенштейна и Дамерау-Левенштейна;
	\item Проведение сравнительного анализа по времени трёх алгоритмов поиска расстояний Дамерау-Левенштейна;
	\item Подготовка отчета о выполненной лабораторной работе.
\end{enumerate}