\chapter*{Заключение}
\addcontentsline{toc}{chapter}{Заключение}

В результате выполнения данной лабораторной работы были рассмотрены однопоточная и многопоточная версии заданного алгоритма обработки разреженной матрицы, упакованной по КРМ-схеме, построены схемы, соответствующие данным алгоритмам, и выполнена их реализация. Можно сделать вывод, что для больших значений Q стоит применять многопоточную реализацию заданного алгоритма. Проведенные замеры времени работы реализаций показали, что на выбранной архитектуре ЭВМ достигается ускорение в 3 раза при числе потоков, кратным числу логических ядер процессора.

В рамках выполнения работы цель достигнута: получен навык организации параллельных вычислений на основе нативных потоков на примере обработки разреженных матриц, представленных с помощью кольцевой КРМ-схемы.

Решены все задачи:
\begin{itemize}
	\item описана КРМ-схема хранения матриц;
	\item разработан и реализован однопоточный алгоритм преобразования верхнетреугольной матрицы к набору матриц, в каждой из которых исключены элементы, большие k, где k от 1 до Q (Q-вход);
	\item разработана и реализована многопоточная версия данного алгоритма;
	\item созданы схемы изучаемых алгоритмов;
	\item определены средства программной реализации;
	\item выполнены замеры процессорного времени работы реализаций алгоритма;
	\item проведен сравнительный анализ по времени работы реализаций алгоритмов;
	\item подготовлен отчет о выполненной лабораторной работе.
\end{itemize}