\chapter*{Введение}
\addcontentsline{toc}{chapter}{Введение}

Одной из задач программирования является ускорение решения вычислительных задач. Один из способов ее решения --- использование параллельных вычислений.

В компьютерной архитектуре многопоточность --- способность центрального процессора (CPU) или одного ядра в многоядерном процессоре одновременно выполнять несколько процессов или потоков, соответствующим образом поддерживаемых операционной системой. Этот подход отличается от многопроцессорности, так как многопоточность процессов и потоков совместно использует ресурсы одного или нескольких ядер: вычислительных блоков, кэш-памяти ЦПУ или буфера перевода с преобразованием (TLB) \cite{multithreading}.

Целью данной лабораторной работы является получение навыка организации параллельных вычислений на основе нативных потоков на примере обработки разреженных матриц, представленных с помощью кольцевой КРМ-схемы. Для достижения поставленной цели требуется решить задачи, представленные ниже.

\begin{enumerate}
	\item Описать КРМ-схему хранения матриц.
	\item Разработать и реализовать однопоточный алгоритм преобразования верхнетреугольной матрицы к набору матриц, в каждой из которых исключены элементы, большие k, где k от 1 до Q (Q-вход).
	\item Разработать и реализовать многопоточную версию данного алгоритма.
	\item Создать схемы изучаемых алгоритмов.
	\item Определить средства программной реализации.
	\item Выполнить замеры процессорного времени работы реализаций алгоритма.
	\item Провести сравнительный анализ по времени работы реализаций алгоритмов.
	\item Подготовить отчет о выполненной лабораторной работе.
\end{enumerate}