\chapter{Технологическая часть}

В данном разделе приведены средства реализации, сведения о модулях программы, листинги кода, тесты.

\section{Средства реализации}

В качестве языка программирования для реализации данной лабора-
торной работы был выбран язык $C++$ \cite{c++}. Данный выбор обусловлен наличием инструментов для реализации принципов многопоточного программирования.

Замеры времени проводились при помощи функции $std::chrono::system\_clock::now(...)$ из библиотеки $chrono$ \cite{time}.

Реализация графического представления замеров времени производилась при помощи языка программирования $Python$ \cite{python}, так как данный язык программирования представляет графическую библиотеку для визуализации данных.

\section{Сведения о модулях программы}

Данная программа разбита на следующие модули:

\begin{itemize}
	\item main.cpp --- файл, содержащий точку входа в программу, в нем происходит вызов алгоритмов;
	\item matrix.cpp и matrix.h --- файлы, содержащие алгоритмы работы с матрицами;
	\item constants.h --- файл, содержащий необходимые константы и объявления;
	\item measurements.cpp и measurements.h --- файл, содержащий функции замеров времени работы алгоритмов;
\end{itemize}

\section{Реализация алгоритмов}

В листингах \ref{lst:alg}--\ref{lst:multi} представлены реализации последовательного и параллельного алгоритмов обработки упакованной разреженной матрицы.
\clearpage
\begin{center}
	\captionsetup{justification=raggedright,singlelinecheck=off}
	\begin{lstlisting}[label=lst:alg,caption=Реализация последовательного алгоритма обработки упакованной разреженной матрицы при фиксированном Q]
		
		void delete_elements_from_matrix(matrix_t *matrix, int k) {
			for (int i = 0; i < matrix->amount;) {
				if (matrix->AN[i] > k) {
					for (int j = 0; j < matrix->amount; ++j) {
						if (matrix->NR[j] == i + 1) {
							matrix->NR[j] = matrix->NR[i];
						}
						if (matrix->NC[j] == i + 1) {
							matrix->NC[j] = matrix->NC[i];
						}
					}
					for (int m = 0; m < matrix->size; ++m) {
						if (matrix->JR[m] == i + 1) {
							if (matrix->NR[i] == i + 1) {
								matrix->JR[m] = 0;
							} else {
								matrix->JR[m] = matrix->NR[i];
							}
						}
						if (matrix->JR[m] > i + 1) {
							--matrix->JR[m];
						}
						if (matrix->JC[m] == i + 1) {
							if (matrix->NC[i] == i + 1) {
								matrix->JC[m] = 0;
							} else {
								matrix->JC[m] = matrix->NC[i];
							}
						}
						if (matrix->JC[m] > i + 1) {
							--matrix->JC[m];
						}
					}
					delete_element(matrix->NR, i, matrix->amount);
					delete_element(matrix->AN, i, matrix->amount);
					delete_element(matrix->NC, i, matrix->amount);
					--matrix->amount;
					for (int j = 0; j < matrix->amount; ++j) {
						if (matrix->NC[j] > i + 1) {
							--matrix->NC[j];
						}
						if (matrix->NR[j] > i + 1) {
							--matrix->NR[j];
						}
					}
				} else {
					++i;
				}
			}
		}
		
	\end{lstlisting} 
\end{center}


\begin{center}
\captionsetup{justification=raggedright,singlelinecheck=off}
\begin{lstlisting}[label=lst:single,caption=Однопоточная реализация алгоритма обработки упакованной разреженной матрицы]

matrix_t *process (matrix_t matrix, int Q) {
	cout << "RESULT:" << endl;
	matrix_t *results = new matrix_t[Q];
	for (int k = 0; k < Q; ++k) {
		results[k].amount = matrix.amount;
		results[k].size = matrix.size;
		results[k].JR = new int[matrix.size];
		results[k].JC = new int[matrix.size];
		results[k].AN = new int[matrix.amount];
		results[k].NR = new int[matrix.amount];
		results[k].NC = new int[matrix.amount];
		
		memcpy(results[k].JR, matrix.JR, matrix.size * sizeof(int));
		memcpy(results[k].JC, matrix.JC, matrix.size * sizeof(int));
		memcpy(results[k].AN, matrix.AN, matrix.amount * sizeof(int));
		memcpy(results[k].NR, matrix.NR, matrix.amount * sizeof(int));
		memcpy(results[k].NC, matrix.NC, matrix.amount * sizeof(int));
		
		delete_elements_from_matrix(&results[k], k + 1);
	}
	return results;
}

\end{lstlisting} 
\end{center}

\begin{center}
\captionsetup{justification=raggedright,singlelinecheck=off}
\begin{lstlisting}[label=lst:multi,caption=Многопоточная реализация алгоритма обработки упакованной разреженной матрицы]

void parallel_process(matrix_t* matrix, std::vector<matrix_t> results, int count_threads, int thread_index, int Q)
{
	cout << endl << "======== THREAD " << thread_index + 1 << " " << "START" << endl;
	
	int step = Q / count_threads;
	int start = thread_index * step;
	
	if (thread_index + 1 == count_threads)
	{
		step += Q - step * count_threads;
	}
	
	for (int k = start; k < start + step; k++)
	{
		results[k].amount = matrix->amount;
		results[k].size = matrix->size;
		results[k].JR = new int[matrix->size];
		results[k].JC = new int[matrix->size];
		results[k].AN = new int[matrix->amount];
		results[k].NR = new int[matrix->amount];
		results[k].NC = new int[matrix->amount];
		
		memcpy(results[k].JR, matrix->JR, matrix->size * sizeof(int));
		memcpy(results[k].JC, matrix->JC, matrix->size * sizeof(int));
		memcpy(results[k].AN, matrix->AN, matrix->amount * sizeof(int));
		memcpy(results[k].NR, matrix->NR, matrix->amount * sizeof(int));
		memcpy(results[k].NC, matrix->NC, matrix->amount * sizeof(int));
		delete_elements_from_matrix(&results[k], k + 1);
		cout << "k = " << k + 1 << endl;
		print_matrix(&results[k]);
	}
	
	cout << endl << "======== THREAD " << thread_index + 1 << " " << "END" << endl;
}

std::vector<matrix_t> multithreading(int count_threads, matrix_t* matrix, int Q)
{
	std::vector<std::thread> threads(count_threads);
	std::vector<matrix_t> results(Q);
	
	for (int i = 0; i < count_threads; i++)
	{
		threads[i] = std::thread(parallel_process, std::ref(matrix), std::ref(results), count_threads, i, Q);
	}
	
	for (int i = 0; i < count_threads; i++)
	{
		threads[i].join();
	}
	
	return results;
}
	    
\end{lstlisting}
\end{center}

\section{Тестирование}

В таблице \ref{tbl:tests} приведены функциональные тесты для реализованного алгоритма.

\begin{table}[h]
	\begin{center}
		\caption{\label{tbl:tests} Функциональные тесты}
		\begin{tabular}{|c|c|}
			\hline
			Входная матрица и Q & Ожидаемый результат \\ 
			\hline
			$[]$ & $[]$\\ 
			\hline
			$\begin{array}{cccc}
				$AN:8, 7, 4, 6, 2, 9$\\
				$NR:2, 3, 1, 5, 4, 6$\\
				$NC:1, 4, 5, 2, 6, 3$\\
				$JR:1, 4, 6$\\
				$JC:1, 2, 3$\\
				$Q = 2$
			\end{array}$
			 &
			$\begin{array}{cccc}
				$k = 1$\\
				$AN:$\\
				$NR:$\\
				$NC:$\\
				$JR:0, 0, 0$\\
				$JC:0, 0, 0$\\
				$k = 2$\\
				$AN:2$\\
				$NR:1$\\
				$NC:1$\\
				$JR:0, 1, 0$\\
				$JC:0, 0, 1$
			\end{array}$
			\\ \hline
		\end{tabular}
	\end{center}
\end{table}

При проведении функционального тестирования, полученные результаты работы программы совпали с ожидаемыми. Таким образом, функциональное тестирование пройдено успешно.

\section*{Вывод}

Были реализованы однопоточная и многопоточная версии заданного алгоритма.