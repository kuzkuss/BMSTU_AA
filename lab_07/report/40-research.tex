\chapter{Исследовательская часть}

В данном разделе приведена постановка эксперимента.

\section{Исследование}

В данной работе, в качестве респондентов, принимали участие следующие студенты:
\begin{itemize}[label=---]
	\item Косарев А.А.;
	\item Никулина А.А.;
	\item Бурлаков И.А.;
	\item Баринов Н.Ю.;
	\item Каландадзе Д.В.
\end{itemize}

Исследуемый признак --- длина рек.

Возможные значения признака (описывают длину):
\begin{itemize}[label=---]
	\item короткая;
	\item средняя;
	\item длинная;
	\item очень длинная;
	\item не очень длинная; 
	\item не очень короткая.
\end{itemize}

\newpage
По результатам опроса была сформирована и занесена в таблицу \ref{tab:time1} обобщённая статистика.

Сокращения, используемые в таблице:
\begin{itemize}[label=---]
	\item К --- короткая;
	\item С --- длинная;
	\item Д --- много;
	\item ОД --- очень длинная;
	\item НОД --- не очень длинная; 
	\item НОК --- не очень короткая.
\end{itemize}


\captionsetup{format=hang, justification=raggedright, 
	singlelinecheck=off}
\begin{table}[h]
	\begin{center}
		\caption{\label{tab:time1}Обобщенная статистика}
		\begin{tabular}{|c|c|c|c|c|c|c|}
		\hline
		Длина реки, км & К & С & Д & ОД & НОД & НОК \\
		\hline
		<1000 & 4 & 0 & 0 & 0 & 0 & 1  \\
		\hline
		>1000 и <3000 & 1 & 0 & 0 & 0 & 3 & 1 \\
		\hline
		>3000 и <4000 & 0 & 2 & 1 & 0 & 1 & 1 \\
		\hline
		>4000 и <5000 & 0 & 1 & 2 & 0 & 0 & 2  \\
		\hline
		>5000 и <6000 & 0 & 0 & 3 & 2 & 0 & 0  \\
		\hline
		>6000 & 0 & 0 & 0 & 5 & 0 & 0  \\
		\hline
		\end{tabular}
	\end{center}
\end{table}

В данной таблице приведены количество голосов, отданных респондентами в пользу истинности разных утверждений. В узлах таблицы расположено количество голосов в пользу высказывания, формируемого по принципу  «длина реки --- тезис» (Например, за истинность высказывания «длина реки менее 1000 км --- короткая» проголосовали четыре человека).

Таблица \ref{tab:time2} содержит нормализованные значения из таблицы \ref{tab:time1}.

\captionsetup{format=hang, justification=raggedright, 
	singlelinecheck=off}
\begin{table}[h]
	\begin{center}
		\caption{\label{tab:time2}Нормализованные значения}
		\begin{tabular}{|c|c|c|c|c|c|c|}
		\hline
		Длина реки, км& М & С & МН & ОМН & НОМН & НОМ \\
		\hline
		<1000 & 0.8 & 0 & 0 & 0 & 0 & 0.2  \\
		\hline
		>1000 и <3000 & 0.2 & 0 & 0 & 0 & 0.6 & 0.2 \\
		\hline
		>3000 и <4000 & 0 & 0.4 & 0.2 & 0 & 0.2 & 0.2 \\
		\hline
		>4000 и <5000 & 0 & 0.2 & 0.4 & 0 & 0 & 0.4  \\
		\hline
		>5000 и <6000 & 0 & 0 & 0.6 & 0.4 & 0 & 0  \\
		\hline
		>6000 & 0 & 0 & 0 & 1 & 0 & 0  \\
		\hline
		\end{tabular}
	\end{center}
\end{table}


\newpage

\section{Функции принадлежности}

Построенные функции принадлежности термам числовых значений признака, описываемого лингвистической переменной, на основе статистической обработки мнений респондентов, выступающих в роли экспертов, приведены на рисунке 4.1.

\img{90mm}{f3}{Функции принадлежности}

\section*{Вывод}

Таким образом, на основании экспертной оценки, были точечно заданы функции принадлежности числового значения указанного признака некоторому термину. Благодаря данным функциям можно с некоторой точностью определить описанную принадлежность.

