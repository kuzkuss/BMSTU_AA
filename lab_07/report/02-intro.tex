
\chapter*{Введение}
\addcontentsline{toc}{chapter}{Введение}

Словарь, как тип данных, применяется везде, где есть связь <<ключ -- значение>> или <<объект -- данные>>. Поиск --- основная задача при использовании словаря. Данная задача решается различными способами, которые дают различную скорость решения.

Цель данной работы: получить навык поиска по словарю при ограничении на значение признака, заданного при помощи лингвистической переменной.

Для достижения цели поставлены следующие задачи:
\begin{itemize}[label=---]
	\item формализовать объект и его признак;
	\item составить анкету для заполнения респондентом;
	\item провести анкетирование респондентов;
	\item описать структуру данных словаря;
	\item предложить и реализовать алгоритм поиска в словаре.
\end{itemize}