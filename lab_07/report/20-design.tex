
\chapter{Конструкторская часть}
В этом разделе будут представлено описание используемых типов данных,
а также схемы алгоритмов поиска в словаре.

\section{Описание используемых типов данных}
При реализации алгоритмов будут использованы следующие типы данных:
\begin{itemize}
	\item словарь -- встроенный тип dict \cite{pythondict} в Python\cite{pythonlang} будет использован в созданном классе Dictionary;
	\item массив ключей -- встроенный тип list \cite{pythonlist} в Python\cite{pythonlang};
	\item длина массива/словаря -- целое число int.
\end{itemize}

\section{Структура разрабатываемой программы}
В данном ПО буде реализован метод структурного программирования, при этом также будет реализован класс Dictionary для работы со словарем.

Взаимодействие с пользователем будет через консоль, будет дана возможность ввода строки для поиска значений в словаре.

\section{Схемы алгоритмов}
На рисунке \ref{img:s1} представлена схема алгоритма поиска в словаре полным перебором.

\img{150mm}{s1}{Схема алгоритма поиска в словаре полным перебором}
\clearpage


\section{Классы эквивалентности при тестировании}
Для тестирования выделены классы эквивалентности, представленные ниже.
\begin{itemize}
	\item некорректный ввод ключа -- пустая строка;
	\item корректный ввод, но ключа нет в словаре -- вывод будет (-1), как знак того, что такого ключа нет словаре;
	\item корректный ввод и ключ есть в словаре -- вывод верного значения.
\end{itemize}

\section*{Вывод}
В данном разделе была построена схема алгоритма, рассматриваемого в лабораторной работе, были описаны классы эквивалентности для тестирования, структура программы.