\chapter{Технологическая часть}

В данном разделе приведены средства реализации, сведения о модулях программы, листинги кода, тесты.

\section{Средства реализации}

В качестве языка программирования для реализации данной лабора-
торной работы был выбран язык Python \cite{python}. Данный выбор обусловлен тем, что он позволяет реализовывать сложные задачи за короткие сроки за счет наличия большого количества подключаемых библиотек и простоты синтаксиса.

Замеры времени проводились при помощи функции process\_time\_ns из библиотеки time \cite{python-time}.

\section{Сведения о модулях программы}

Данная программа разбита на следующие модули:

\begin{itemize}
	\item main.py - файл, содержащий точку входа в программу, в нем происходит вызов алгоритмов;
	\item multiplication.py - файл, содержащий алгоритмы умножения матриц;
	\item matrix.py - файл, содержащий функции работы с матрицами;
	\item measurements.py - файл, содержащий функции замеров времени работы алгоритмов;
\end{itemize}

\section{Листинги кода}

В листингах \ref{lst:classic}--\ref{lst:optimized_winograd} представлены реализации алгоритмов умножения матриц: классического, Винограда и Винограда с оптимизациями.
\clearpage
\begin{center}
\captionsetup{justification=raggedright,singlelinecheck=off}
\begin{lstlisting}[label=lst:classic,caption=Классический алгоритм умножения матриц]
def classic_multiplication(mtrx1, mtrx2):
	n, l, m = len(mtrx1), len(mtrx1[0]), len(mtrx2[0])
	
	res_mtrx = [[0] * m for _ in range(n)]
	
	for i in range(n):
		for j in range(m):
			for k in range(l):
				res_mtrx[i][j] += mtrx1[i][k] * mtrx2[k][j]
	
	return res_mtrx

\end{lstlisting} 
\end{center}

\begin{center}
\captionsetup{justification=raggedright,singlelinecheck=off}
\begin{lstlisting}[label=lst:winograd,caption=Алгоритм Винограда]
def winograd_multiplication(mtrx1, mtrx2):
	n, l, m = len(mtrx1), len(mtrx1[0]), len(mtrx2[0])
	
	res_mtrx = [[0] * m for _ in range(n)]
	
	tmp_mtrx1 = precompute_rows(mtrx1, n, l)
	tmp_mtrx2 = precompute_cols(mtrx2, l, m)
	
	for i in range(n):
		for j in range(m):
			res_mtrx[i][j] = -tmp_mtrx1[i] - tmp_mtrx2[j]
			for k in range(l // 2):
				res_mtrx[i][j] = res_mtrx[i][j] + (mtrx1[i][k * 2] + mtrx2[k * 2 + 1][j]) * (mtrx1[i][k * 2 + 1] + mtrx2[k * 2][j])
	
	if l % 2 != 0:
		for i in range(n):
			for j in range(m):
				res_mtrx[i][j] = res_mtrx[i][j] + mtrx1[i][l - 1] * mtrx2[l - 1][j]
	return res_mtrx

def precompute_rows(mtrx, n, m):
	res_vec = [0] * n
	
	for i in range(n):
		for j in range(m // 2):
			res_vec[i] = res_vec[i] + mtrx[i][j * 2] * mtrx[i][j * 2 + 1]
	
	return res_vec

def precompute_cols(mtrx, n, m):
	res_vec = [0] * m
	
	for i in range(m):
		for j in range(n // 2):
			res_vec[i] = res_vec[i] + mtrx[j * 2][i] * mtrx[j * 2 + 1][i]
	
	return res_vec
    
\end{lstlisting}
\end{center}

\begin{center}
\captionsetup{justification=raggedright,singlelinecheck=off}
\begin{lstlisting}[label=lst:optimized_winograd,caption=Оптимизированный алгоритм Винограда]
def optimized_winograd_multiplication(mtrx1, mtrx2):
	n, l, m = len(mtrx1), len(mtrx1[0]), len(mtrx2[0])
	
	res_mtrx = [[0] * m for _ in range(n)]
	
	tmp_mtrx1 = optimized_precompute_rows(mtrx1, n, l)
	tmp_mtrx2 = optimized_precompute_cols(mtrx2, l , m)
	
	for i in range(n):
		for j in range(m):
			temp = -tmp_mtrx1[i] - tmp_mtrx2[j]
			for k in range(l // 2):
				temp += (mtrx1[i][k << 1] + mtrx2[(k << 1) + 1][j]) * (mtrx1[i][(k << 1) + 1] + mtrx2[k << 1][j])
			res_mtrx[i][j] = temp
	
	if l % 2 != 0:
		for i in range(n):
			for j in range(m):
				res_mtrx[i][j] += mtrx1[i][l - 1] * mtrx2[l - 1][j]
	return res_mtrx

def optimized_precompute_rows(mtrx, n, m):
	res_vec = [0] * n
	
	for i in range(n):
		for j in range(m // 2):
			res_vec[i] += mtrx[i][j << 1] * mtrx[i][(j << 1) + 1]
	
	return res_vec

def optimized_precompute_cols(mtrx, n, m):
	res_vec = [0] * m
	
	for i in range(m):
		for j in range(n // 2):
			res_vec[i] += mtrx[j << 1][i] * mtrx[(j << 1) + 1][i]
		
	return res_vec

\end{lstlisting}
\end{center}

\section{Тестирование}

В таблице \ref{tbl:tests} приведены функциональные тесты для алгоритмов умножения матриц.


\begin{table}[h]
	\begin{center}
		\begin{threeparttable}
			\captionsetup{justification=raggedright,singlelinecheck=off}
			\caption{\label{tbl:tests} Тесты}
			\begin{tabular}{|c@{\hspace{7mm}}|c@{\hspace{7mm}}|c@{\hspace{7mm}}|c@{\hspace{7mm}}|c@{\hspace{7mm}}|c@{\hspace{7mm}}|}
				\hline
				Матрица 1 & Матрица 2 & Ожидаемый результат \\ 
				\hline
				$\begin{pmatrix}
					&
				\end{pmatrix}$ &
				$\begin{pmatrix}
					&
				\end{pmatrix}$ &
				Сообщение об ошибке \\ \hline
				
				$\begin{pmatrix}
					1 & 1
				\end{pmatrix}$ &
				$\begin{pmatrix}
					1 & 1
				\end{pmatrix}$ &
				Сообщение об ошибке \\ \hline
				
				$\begin{pmatrix}
					1
				\end{pmatrix}$ &
				$\begin{pmatrix}
					1
				\end{pmatrix}$ &
				$\begin{pmatrix}
					1
				\end{pmatrix}$ \\ \hline
			
				$\begin{pmatrix}
					1 \\
					2
				\end{pmatrix}$ &
				$\begin{pmatrix}
					1 & 2
				\end{pmatrix}$ &
				$\begin{pmatrix}
					1 & 2 \\
					2 & 4
				\end{pmatrix}$ \\ \hline
				
				$\begin{pmatrix}
					1 & 2 & 3 \\
					1 & 2 & 3 \\
					1 & 2 & 3
				\end{pmatrix}$ &
				$\begin{pmatrix}
					3 & 2 & 1 \\
					3 & 2 & 1 \\
					3 & 2 & 1
				\end{pmatrix}$ &
				$\begin{pmatrix}
					18 & 12 & 6 \\
					18 & 12 & 6 \\
					18 & 12 & 6
				\end{pmatrix}$ \\ \hline
				
			\end{tabular}
		\end{threeparttable}
	\end{center}
	
\end{table}

При проведении функционального тестирования, полученные результаты работы программы совпали с ожидаемыми. Таким образом, функциональное тестирование пройдено успешно.

\section*{Вывод}

Был реализован классический алгоритм умножения матриц, алгоритм Винограда и оптимизированный алгоритм Винограда.