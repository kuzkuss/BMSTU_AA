\chapter{Аналитическая часть}

В данном разделе представлено теоретическое описание алгоритмов умножения матриц: классического, Винограда и Винограда с оптимизациями.

\section{Классический алгоритм умножения матриц}

По определению умножения матриц для $n \times m$ матрицы A:

\begin{equation}
	A = \left(
	\begin{array}{cccc}
		a_{11} & a_{12} & \ldots & a_{1m} \\
		a_{21} & a_{22} & \ldots & a_{2m} \\
		\vdots & \vdots & \ddots & \vdots \\
		a_{n1} & a_{n2} & \ldots & a_{nm}
	\end{array}
	\right)
	\label{eq:ref1}
\end{equation}

и $m \times p$ матрицы B:

\begin{equation}
	B = \left(
	\begin{array}{cccc}
		a_{11} & a_{12} & \ldots & a_{1p} \\
		a_{21} & a_{22} & \ldots & a_{2p} \\
		\vdots & \vdots & \ddots & \vdots \\
		a_{m1} & a_{m2} & \ldots & a_{mp}
	\end{array}
	\right)
	\label{eq:ref2}
\end{equation}

произведением $C=AB$ является $n \times p$ матрица, состоящая из элементов:
\begin{equation}
	\begin{array}{cc}
		c_{ij} = \sum\limits_{k=1}^m a_{ik}b_{kj} & (i=1,2,\dots n; j=1,2,\dots p)
	\end{array}
	\label{eq:ref3}
\end{equation}

Для того, чтобы умножить матрицу $A$ размером $n \times m$ и матрицу $B$ размером $m \times p$, нужно взять $i$-ю строку ($i=1,2,\dots n$) в первой матрице и $j$-й ($j=1,2,\dots p$) столбец во второй, перемножить их поэлементно, сложить полученные произведения и записать результат в $i$-й, $j$-й элемент результирующей матрицы.

\section{Алгоритм Винограда}

Каждый элемент результирующей матрицы представляет собой скалярное произведение соответствующих строки и столбца.

Рассмотрим два вектора:
\begin{equation}
	A = (a_{1},..., a_{n})
	\label{eq:ref4}
\end{equation}
\begin{equation}
	B = (b_{1},..., b_{n})
	\label{eq:ref5}
\end{equation}
Их скалярное произведение равно:
\begin{equation}
	\begin{array}{cc}
		AB = \sum\limits_{1}^n a_{i}b_{i}
	\end{array}
	\label{eq:ref6}
\end{equation}
\begin{equation}
	\begin{array}{cc}
		AB = \sum\limits_{1}^{n/2} (a_{2i} + b_{2i+1})(a_{2i+1} + b_{2i})-t
	\end{array}
	\label{eq:ref7}
\end{equation}
\begin{equation}
	\begin{array}{cc}
		t = \sum\limits_{1}^{n/2} a_{2i}a_{2i+1} + \sum\limits_{1}^{n/2} b_{2i}b_{2i+1}
	\end{array}
	\label{eq:ref8}
\end{equation}

В выражении \ref{eq:ref7} и \ref{eq:ref8} требуется большее
число вычислений, чем в первом, но оно позволяет произвести предварительную обработку. Выражения $a_{2i}a_{2i+1}$ и $b_{2i}b_{2i+1}$ для $i=1,2,\dots n/2$ можно вычислить заранее для каждой соответствующей строки и столбца. Это позволяет уменьшить число умножений \cite{winograd}.

\section{Оптимизированный алгоритм Винограда}

Оптимизированный алгоритм Винограда схематично представляет собой обычный алгоритм Винограда, за исключением следующих оптимизаций:

\begin{itemize}
	\item замена операции $x=x+k$ на $x+=k$;
	\item замена умножения на 2 на побитовый сдвиг;
	\item предвычисление некоторых слагаемых для алгоритма.
\end{itemize}


\section*{Вывод}

В данном разделе были описаны алгоритмы умножения матриц: классический, Винограда и Винограда с оптимизациямми.