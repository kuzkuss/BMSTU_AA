\chapter*{Введение}
\addcontentsline{toc}{chapter}{Введение}

В современном мире приходится работать с большими
массивами данных. Для упорядочивания данных и последующей их
обработке проводится специальная операция по представлению
данных в порядке увеличения (или уменьшения) их значения. Данную
операцию называют сортировкой. В данной работе приведено
сравнение трех популярных методов сортировки данных \cite{winograd}.

С матрицами можно выполнять стандартные алгебраические операции: сложение, вычитание, умножение, деление.

В данной лабораторной работе будут рассмотрены алгоритмы умножения матриц.
Умножение матриц --- одна из основных операций над матрицами. Матрица, получаемая в результате операции умножения, называется произведением матриц.
Две матрицы могут быть перемножены, если число столбцов первой матрицы равно числу строк второй матрицы.

Умножение матриц является основным инструментом линейной алгебры и имеет многочисленные применения в математике, физике, программировании. Одним из самых эффективных по времени алгоритмов умножения матриц является алгоритм Винограда, имеющий асимптотическую сложность $O(n^{2,3755})$. Также существуют улучшения
этого алгоритма \cite{winograd}.

Целью данной лабораторной работы является изучение алгоритмов умножения матриц. Для достижения поставленной цели требуется решить следующие задачи:

\begin{enumerate}
	\item Реализация классического алгоритма умножения матриц;
	\item Реализация алгоритма Винограда умножения матриц;
	\item Реализация алгоритма Винограда умножения матриц с оптимизациями;
	\item Создание схем изучаемых алгоритмов;
	\item Оценка трудоёмкости алгоритмов;
	\item Определение средств программной реализации алгоритмов;
	\item Выполнение замеров процессорного времени работы реализаций алгоритмов умножения матриц;
	\item Проведение сравнительного анализа по времени алгоритмов умножения матриц;
	\item Подготовка отчета о выполненной лабораторной работе.
\end{enumerate}