\chapter*{Введение}
\addcontentsline{toc}{chapter}{Введение}

Существуют задачи, в которых разные алгоритмы обрабатывают один и тот же набор данных друг за другом. При этом, может стоять задача обработки большого объема данных. Для ускорения решения таких вычислительных задач используется конвейерная обработка.

Идея конвейерной обработки \cite{idea} заключается в выделении отдельных этапов выполнения общей операции. Каждый этап, выполнив свою работу, передает результат следующему, одновременно принимая новую порцию входных данных. Отдельный этап называют лентой. При таком способе организации вычислений увеличивается скорость обработки за счет совмещения прежде разнесенных во времени операций.

В многопоточном программировании конвейерная обработка реализуется следующим образом: под каждую ленту конвейера выделяется отдельный поток. Выделенные потоки работают асинхронно.

Целью данной лабораторной работы является получение навыка организации асинхронного взаимодействия между потоками на примере моделирования конвейера обработки разреженных матриц, представленных с помощью кольцевой КРМ-схемы. Для достижения поставленной цели требуется выполнить следующие задачи.

\begin{enumerate}
	\item Описать КРМ-схему хранения матриц.
	\item Разработать и реализовать последовательный алгоритм.
	\item Выделить алгоритмы, выполняемые на отдельных лентах конвейера.
	\item Разработать и реализовать конвейерную версию данного алгоритма.
	\item Создать схемы изучаемых алгоритмов.
	\item Определить средства программной реализации.
	\item Выполнить замеры процессорного времени работы реализаций алгоритма.
	\item Провести сравнительный анализ по времени работы реализаций алгоритмов.
	\item Подготовить отчет о выполненной лабораторной работе.
\end{enumerate}