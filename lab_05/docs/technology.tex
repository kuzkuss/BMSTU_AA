\chapter{Технологическая часть}

В данном разделе приведены средства реализации, сведения о модулях программы, листинги кода, тесты.

\section{Средства реализации}

В качестве языка программирования для реализации данной лабора-
торной работы был выбран язык $C++$ \cite{c++}. Данный выбор обусловлен наличием инструментов для реализации принципов многопоточного программирования.

Замеры времени проводились при помощи функции $std::chrono::system\_clock::now(...)$ из библиотеки $chrono$ \cite{time}.

Реализация графического представления замеров времени производилась при помощи языка программирования $Python$ \cite{python}, так как данный язык программирования представляет графическую библиотеку для визуализации данных.

\section{Сведения о модулях программы}

Данная программа разбита на следующие модули:

\begin{itemize}
	\item main.cpp --- файл, содержащий точку входа в программу, в нем происходит вызов алгоритмов;
	\item matrix.cpp и matrix.h --- файлы, содержащие алгоритмы работы с матрицами;
	\item constants.h --- файл, содержащий необходимые константы и объявления;
	\item conveyor.cpp и conveyor.h --- файлы, содержащие алгоритмы работы конвейера;
	\item measurements.cpp и measurements.h --- файлы, содержащий функции замеров времени работы алгоритмов;
\end{itemize}

\section{Реализация алгоритмов}

В листингах \ref{lst:linear}--\ref{lst:conveyor} представлены реализации последовательного и конвейерного алгоритмов обработки матриц. В листингах \ref{lst:creation}--\ref{lst:output} представлены алгоритмы работы с матрицами, выполняемые на отдельных лентах конвейера.
\clearpage
\begin{center}
	\captionsetup{justification=raggedright,singlelinecheck=off}
	\begin{lstlisting}[label=lst:linear,caption=Реализация последовательного алгоритма обработки матриц]
		
void linear_log(matrix_t *mtrx, const int belt, const int task)
{
	std::chrono::time_point<std::chrono::system_clock> start, end;
	double result = 0;
	
	if ( belt == FIRST )
	{
		start = std::chrono::system_clock::now();
		create_matrix(mtrx);
		end = std::chrono::system_clock::now();
		print_matrix(mtrx, task);
	}
	else if ( belt == SECOND )
	{
		start = std::chrono::system_clock::now();
		delete_elements_from_matrix(mtrx, 5);
		end = std::chrono::system_clock::now();
	}
	else if ( belt == THIRD )
	{
		start = std::chrono::system_clock::now();
		print_matrix(mtrx, task);
		end = std::chrono::system_clock::now();
		free_matrix(mtrx);
	}
	
	result=(std::chrono::duration_cast <std::chrono::nanoseconds>(end - start).count()) / IN_SEC;
	printf("Tape: %d  Task: %d  Start Time: %.6f  End Time: %.6f\n", belt, task, cur_time, cur_time + result);
	cur_time += result;
}


void linear_mode(int size, int n)
{
	std::queue<matrix_t> q1;
	std::queue<matrix_t> q2;
	std::queue<matrix_t> q3;
	
	for ( int i = 0; i < n; ++i )
	{
		matrix_t mtrx;
		init_matrix(&mtrx);
		
		mtrx.size = size;
		
		q1.push(mtrx);
	}
	
	for ( int i = 0; i < n; ++i )
	{
		matrix_t mtrx = q1.front();
		linear_log(&mtrx, 1, i + 1);
		q2.push(mtrx);
		q1.pop();
		
		mtrx = q2.front();
		linear_log(&mtrx, 2, i + 1);
		q3.push(mtrx);
		q2.pop();
		
		mtrx = q3.front();
		linear_log(&mtrx, 3, i + 1);
		q3.pop();
	}
}
	\end{lstlisting} 
\end{center}


\begin{center}
\captionsetup{justification=raggedright,singlelinecheck=off}
\begin{lstlisting}[label=lst:conveyor,caption=Реализация конвейерного алгоритма обработки матриц]

void parallel_log(matrix_t *mtrx, const int belt, const int task)
{
	std::chrono::time_point<std::chrono::system_clock> start, end;
	double result = 0, start_time;
	
	if ( belt == FIRST )
	{
		start = std::chrono::system_clock::now();
		create_matrix(mtrx);
		end = std::chrono::system_clock::now();
		result = (std::chrono::duration_cast <std::chrono::nanoseconds>(end - start).count()) / IN_SEC;
		
		start_time = cur_time1[task - 1];
		cur_time1[task] = start_time + result;
		cur_time2[task - 1] = cur_time1[task];
		print_matrix(mtrx, task);
	}
	else if ( belt == SECOND )
	{
		start = std::chrono::system_clock::now();
		delete_elements_from_matrix(mtrx, 5);
		end = std::chrono::system_clock::now();
		result = (std::chrono::duration_cast <std::chrono::nanoseconds>(end - start).count()) / IN_SEC;
		
		start_time = cur_time2[task - 1];
		cur_time2[task] = start_time + result;
		cur_time3[task - 1] = cur_time2[task];
	}
	else if ( belt == THIRD )
	{
		start = std::chrono::system_clock::now();
		print_matrix(mtrx, task);
		end = std::chrono::system_clock::now();
		result = (std::chrono::duration_cast <std::chrono::nanoseconds>(end - start).count()) / IN_SEC;
		
		start_time = cur_time3[task - 1];
		cur_time3[task] = start_time + result;
		free_matrix(mtrx);
	}
}

void first_belt(std::queue<matrix_t>& q1, std::queue<matrix_t>& q2)
{
	int task = 0;
	std::mutex m;
	
	while ( !q1.empty())
	{
		m.lock();
		matrix_t mtrx = q1.front();
		m.unlock();
		
		parallel_log(&mtrx, 1, ++task);
		
		m.lock();
		q2.push(mtrx);
		q1.pop();
		m.unlock();
	}
}


void second_belt(std::queue<matrix_t>& q1, std::queue<matrix_t>& q2, std::queue<matrix_t>& q3)
{
	int task = 0;
	std::mutex m;
	
	do
	{
		if ( !q2.empty() )
		{
			m.lock();
			matrix_t mtrx = q2.front();
			m.unlock();
			
			parallel_log(&mtrx, 2, ++task);
			
			m.lock();
			q3.push(mtrx);
			q2.pop();
			m.unlock();
		}
	} while ( !q1.empty() || !q2.empty() );
}


void third_belt(std::queue<matrix_t>& q1, std::queue<matrix_t>& q2, std::queue<matrix_t>& q3)
{
	int task = 0;
	std::mutex m;
	
	do
	{
		if ( !q3.empty() )
		{
			m.lock();
			matrix_t mtrx = q3.front();
			m.unlock();
			
			parallel_log(&mtrx, 3, ++task);
			
			m.lock();
			q3.pop();
			m.unlock();
		}
	} while ( !q1.empty() || !q2.empty() || !q3.empty() );
}


void parallel_mode(int size, int n)
{
	std::queue<matrix_t> q1;
	std::queue<matrix_t> q2;
	std::queue<matrix_t> q3;
	
	for ( int i = 0; i < n; ++i )
	{
		matrix_t mtrx;
		init_matrix(&mtrx);
		
		mtrx.size = size;
		
		q1.push(mtrx);
	}
	
	for ( int i = 0; i < n + 1; ++i )
	{
		cur_time1.push_back(0);
		cur_time2.push_back(0);
		cur_time3.push_back(0);
	}
	
	std::vector<std::thread> threads(COUNT_THREADS);
	
	threads[0] = std::thread(first_belt, std::ref(q1), std::ref(q2));
	threads[1] = std::thread(second_belt, std::ref(q1), std::ref(q2), std::ref(q3));
	threads[2] = std::thread(third_belt, std::ref(q1), std::ref(q2), std::ref(q3));
	
	for ( int i = 0; i < COUNT_THREADS; ++i )
	{
		threads[i].join();
	}
	
	cout << endl;
	for ( int i = 0; i < n; ++i )
	{
		printf("Tape: 1  Task: %d  Start Time: %.6f  End Time: %.6f\n", i + 1, cur_time1[i], cur_time1[i + 1]);
	}
	for ( int i = 0; i < n; ++i )
	{
		printf("Tape: 2  Task: %d  Start Time: %.6f  End Time: %.6f\n", i + 1, cur_time2[i], cur_time2[i + 1]);
	}
	for ( int i = 0; i < n; ++i )
	{
		printf("Tape: 3  Task: %d  Start Time: %.6f  End Time: %.6f\n", i + 1, cur_time3[i], cur_time3[i + 1]);
	}
}

\end{lstlisting} 
\end{center}

\begin{center}
\captionsetup{justification=raggedright,singlelinecheck=off}
\begin{lstlisting}[label=lst:creation,caption=Алгоритм генерации верхнетреугольной упакованной в КРМ-схему хранения матрицы]

void create_matrix(matrix_t* matrix)
{
	srand(time(0));
	
	matrix->amount = (1 + matrix->size) * matrix->size / 2;
	
	matrix->JR = new int[matrix->size];
	matrix->JC = new int[matrix->size];
	matrix->AN = new int[matrix->amount];
	matrix->NR = new int[matrix->amount];
	matrix->NC = new int[matrix->amount];
	
	int idx = 0;
	for (int i = 0; i < matrix->size; ++i) {
		for (int j = i; j < matrix->size; ++j) {
			int tmp;
			tmp = rand() % MAX_NUM + 1;
			
			matrix->AN[idx] = tmp;
			if (j < matrix->size - 1) {
				matrix->NR[idx] = idx + 2;
			} else {
				matrix->NR[idx] = idx - j + i + 1;
			}
			if (i < j) {
				matrix->NC[idx] = idx + matrix->size - i;
			} else {
				matrix->NC[idx] = idx - matrix->size * i + (1 + i) * i / 2 + 1;
				matrix->JR[i] = idx + 1;
			}
			
			if (i == 0) {
				matrix->JC[j] = idx + 1;
			}
			++idx;
		}
	}
}
	    
\end{lstlisting}
\end{center}

\begin{center}
	\captionsetup{justification=raggedright,singlelinecheck=off}
	\begin{lstlisting}[label=lst:deletion,caption=Алгоритм удаления элементов из матрицы меньших или равных $q$]
		
void delete_element(int *arr, int idx, int n) {
	for (int i = idx; i < n - 1; ++i) {
		arr[i] = arr[i + 1];
	}
}

void delete_elements_from_matrix(matrix_t *matrix, int q) {
	for (int i = 0; i < matrix->amount;) {
		if (matrix->AN[i] <= q) {
			for (int j = 0; j < matrix->amount; ++j) {
				if (matrix->NR[j] == i + 1) {
					matrix->NR[j] = matrix->NR[i];
				}
				if (matrix->NC[j] == i + 1) {
					matrix->NC[j] = matrix->NC[i];
				}
			}
			for (int m = 0; m < matrix->size; ++m) {
				if (matrix->JR[m] == i + 1) {
					if (matrix->NR[i] == i + 1) {
						matrix->JR[m] = 0;
					} else {
						matrix->JR[m] = matrix->NR[i];
					}
				}
				if (matrix->JR[m] > i + 1) {
					--matrix->JR[m];
				}
				if (matrix->JC[m] == i + 1) {
					if (matrix->NC[i] == i + 1) {
						matrix->JC[m] = 0;
					} else {
						matrix->JC[m] = matrix->NC[i];
					}
				}
				if (matrix->JC[m] > i + 1) {
					--matrix->JC[m];
				}
			}
			delete_element(matrix->NR, i, matrix->amount);
			delete_element(matrix->AN, i, matrix->amount);
			delete_element(matrix->NC, i, matrix->amount);
			--matrix->amount;
			for (int j = 0; j < matrix->amount; ++j) {
				if (matrix->NC[j] > i + 1) {
					--matrix->NC[j];
				}
				if (matrix->NR[j] > i + 1) {
					--matrix->NR[j];
				}
			}
		} else {
			++i;
		}
	}
}
		
	\end{lstlisting}
\end{center}

\begin{center}
	\captionsetup{justification=raggedright,singlelinecheck=off}
	\begin{lstlisting}[label=lst:output,caption=Алгоритм вывода матрицы в файл]
		
		void print_matrix(matrix_t* matrix, int num)
		{
			ofstream fout("matrix" + to_string(num) + ".txt", ios_base::app);
			fout << "AN:";
			for (int i = 0; i < matrix->amount; ++i) {
				fout << matrix->AN[i] << " ";
			}
			fout << endl;
			
			fout << "NR:";
			for (int i = 0; i < matrix->amount; ++i) {
				fout << matrix->NR[i] << " ";
			}
			fout << endl;
			
			fout << "NC:";
			for (int i = 0; i < matrix->amount; ++i) {
				fout << matrix->NC[i] << " ";
			}
			fout << endl;
			
			fout << "JR:";
			for (int i = 0; i < matrix->size; ++i) {
				fout << matrix->JR[i] << " ";
			}
			fout << endl;
			
			fout << "JC:";
			for (int i = 0; i < matrix->size; ++i) {
				fout << matrix->JC[i] << " ";
			}
			fout << endl;
			fout << endl;
			fout << endl;
			
			fout.close();
		}
		
	\end{lstlisting}
\end{center}

\section{Тестирование}

В таблице \ref{tbl:tests} приведены функциональные тесты для реализованных алгоритмов.

\begin{table}[h]
	\begin{center}
		\caption{\label{tbl:tests} Функциональные тесты}
		\begin{tabular}{|c|c|}
			\hline
			Входные матрицы ($k$ = 5, матрица одна) & Ожидаемый результат \\ 
			\hline
			$[]$ & $[]$\\ 
			\hline
			$\begin{array}{cccc}
				$AN:2 9 6 1 2 7 1 2 10 3 3 6 5 7 6$\\
				$NR:2 3 4 5 1 7 8 9 6 11 12 10 14 13 15 $\\
				$NC:1 6 7 8 9 2 10 11 12 3 13 14 4 15 5$\\
				$JR:1 6 10 13 15$\\
				$JC:1 2 3 4 5$\\
			\end{array}$
			 &
			$\begin{array}{cccc}
				$AN:9 6 7 10 6 7 6$\\
				$NR:2 1 4 3 5 6 7$\\
				$NC:3 2 1 5 6 7 4$\\
				$JR:1 3 5 6 7$\\
				$JC:0 1 2 0 4$
			\end{array}$
			\\ \hline
		\end{tabular}
	\end{center}
\end{table}

При проведении функционального тестирования, полученные результаты работы программы совпали с ожидаемыми. Таким образом, функциональное тестирование пройдено успешно реализациями двух алгоритмов обработки --- последовательной и конвейерной.

\section*{Вывод}

Были реализованы последовательный и конвейерный алгоритмы обработки матриц.